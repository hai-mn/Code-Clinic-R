% Options for packages loaded elsewhere
\PassOptionsToPackage{unicode}{hyperref}
\PassOptionsToPackage{hyphens}{url}
%
\documentclass[
  11pt,
]{article}
\usepackage{lmodern}
\usepackage{amsmath}
\usepackage{ifxetex,ifluatex}
\ifnum 0\ifxetex 1\fi\ifluatex 1\fi=0 % if pdftex
  \usepackage[T1]{fontenc}
  \usepackage[utf8]{inputenc}
  \usepackage{textcomp} % provide euro and other symbols
  \usepackage{amssymb}
\else % if luatex or xetex
  \usepackage{unicode-math}
  \defaultfontfeatures{Scale=MatchLowercase}
  \defaultfontfeatures[\rmfamily]{Ligatures=TeX,Scale=1}
\fi
% Use upquote if available, for straight quotes in verbatim environments
\IfFileExists{upquote.sty}{\usepackage{upquote}}{}
\IfFileExists{microtype.sty}{% use microtype if available
  \usepackage[]{microtype}
  \UseMicrotypeSet[protrusion]{basicmath} % disable protrusion for tt fonts
}{}
\makeatletter
\@ifundefined{KOMAClassName}{% if non-KOMA class
  \IfFileExists{parskip.sty}{%
    \usepackage{parskip}
  }{% else
    \setlength{\parindent}{0pt}
    \setlength{\parskip}{6pt plus 2pt minus 1pt}}
}{% if KOMA class
  \KOMAoptions{parskip=half}}
\makeatother
\usepackage{xcolor}
\IfFileExists{xurl.sty}{\usepackage{xurl}}{} % add URL line breaks if available
\IfFileExists{bookmark.sty}{\usepackage{bookmark}}{\usepackage{hyperref}}
\hypersetup{
  pdftitle={Summarizing The Weather},
  hidelinks,
  pdfcreator={LaTeX via pandoc}}
\urlstyle{same} % disable monospaced font for URLs
\usepackage[margin=1in]{geometry}
\usepackage{color}
\usepackage{fancyvrb}
\newcommand{\VerbBar}{|}
\newcommand{\VERB}{\Verb[commandchars=\\\{\}]}
\DefineVerbatimEnvironment{Highlighting}{Verbatim}{commandchars=\\\{\}}
% Add ',fontsize=\small' for more characters per line
\usepackage{framed}
\definecolor{shadecolor}{RGB}{248,248,248}
\newenvironment{Shaded}{\begin{snugshade}}{\end{snugshade}}
\newcommand{\AlertTok}[1]{\textcolor[rgb]{0.94,0.16,0.16}{#1}}
\newcommand{\AnnotationTok}[1]{\textcolor[rgb]{0.56,0.35,0.01}{\textbf{\textit{#1}}}}
\newcommand{\AttributeTok}[1]{\textcolor[rgb]{0.77,0.63,0.00}{#1}}
\newcommand{\BaseNTok}[1]{\textcolor[rgb]{0.00,0.00,0.81}{#1}}
\newcommand{\BuiltInTok}[1]{#1}
\newcommand{\CharTok}[1]{\textcolor[rgb]{0.31,0.60,0.02}{#1}}
\newcommand{\CommentTok}[1]{\textcolor[rgb]{0.56,0.35,0.01}{\textit{#1}}}
\newcommand{\CommentVarTok}[1]{\textcolor[rgb]{0.56,0.35,0.01}{\textbf{\textit{#1}}}}
\newcommand{\ConstantTok}[1]{\textcolor[rgb]{0.00,0.00,0.00}{#1}}
\newcommand{\ControlFlowTok}[1]{\textcolor[rgb]{0.13,0.29,0.53}{\textbf{#1}}}
\newcommand{\DataTypeTok}[1]{\textcolor[rgb]{0.13,0.29,0.53}{#1}}
\newcommand{\DecValTok}[1]{\textcolor[rgb]{0.00,0.00,0.81}{#1}}
\newcommand{\DocumentationTok}[1]{\textcolor[rgb]{0.56,0.35,0.01}{\textbf{\textit{#1}}}}
\newcommand{\ErrorTok}[1]{\textcolor[rgb]{0.64,0.00,0.00}{\textbf{#1}}}
\newcommand{\ExtensionTok}[1]{#1}
\newcommand{\FloatTok}[1]{\textcolor[rgb]{0.00,0.00,0.81}{#1}}
\newcommand{\FunctionTok}[1]{\textcolor[rgb]{0.00,0.00,0.00}{#1}}
\newcommand{\ImportTok}[1]{#1}
\newcommand{\InformationTok}[1]{\textcolor[rgb]{0.56,0.35,0.01}{\textbf{\textit{#1}}}}
\newcommand{\KeywordTok}[1]{\textcolor[rgb]{0.13,0.29,0.53}{\textbf{#1}}}
\newcommand{\NormalTok}[1]{#1}
\newcommand{\OperatorTok}[1]{\textcolor[rgb]{0.81,0.36,0.00}{\textbf{#1}}}
\newcommand{\OtherTok}[1]{\textcolor[rgb]{0.56,0.35,0.01}{#1}}
\newcommand{\PreprocessorTok}[1]{\textcolor[rgb]{0.56,0.35,0.01}{\textit{#1}}}
\newcommand{\RegionMarkerTok}[1]{#1}
\newcommand{\SpecialCharTok}[1]{\textcolor[rgb]{0.00,0.00,0.00}{#1}}
\newcommand{\SpecialStringTok}[1]{\textcolor[rgb]{0.31,0.60,0.02}{#1}}
\newcommand{\StringTok}[1]{\textcolor[rgb]{0.31,0.60,0.02}{#1}}
\newcommand{\VariableTok}[1]{\textcolor[rgb]{0.00,0.00,0.00}{#1}}
\newcommand{\VerbatimStringTok}[1]{\textcolor[rgb]{0.31,0.60,0.02}{#1}}
\newcommand{\WarningTok}[1]{\textcolor[rgb]{0.56,0.35,0.01}{\textbf{\textit{#1}}}}
\usepackage{graphicx}
\makeatletter
\def\maxwidth{\ifdim\Gin@nat@width>\linewidth\linewidth\else\Gin@nat@width\fi}
\def\maxheight{\ifdim\Gin@nat@height>\textheight\textheight\else\Gin@nat@height\fi}
\makeatother
% Scale images if necessary, so that they will not overflow the page
% margins by default, and it is still possible to overwrite the defaults
% using explicit options in \includegraphics[width, height, ...]{}
\setkeys{Gin}{width=\maxwidth,height=\maxheight,keepaspectratio}
% Set default figure placement to htbp
\makeatletter
\def\fps@figure{htbp}
\makeatother
\setlength{\emergencystretch}{3em} % prevent overfull lines
\providecommand{\tightlist}{%
  \setlength{\itemsep}{0pt}\setlength{\parskip}{0pt}}
\setcounter{secnumdepth}{5}
\usepackage{booktabs}
\usepackage{longtable}
\usepackage{array}
\usepackage{multirow}
\usepackage{wrapfig}
\usepackage{float}
\usepackage{colortbl}
\usepackage{pdflscape}
\usepackage{tabu}
\usepackage{threeparttable}
\usepackage{threeparttablex}
\usepackage[normalem]{ulem}
\usepackage{makecell}
\usepackage{xcolor}
\ifluatex
  \usepackage{selnolig}  % disable illegal ligatures
\fi

\title{Summarizing The Weather}
\author{}
\date{\vspace{-2.5em}4/23/2021}

\begin{document}
\maketitle

{
\setcounter{tocdepth}{2}
\tableofcontents
}
\hypertarget{setup-and-libraries}{%
\section{Setup and Libraries}\label{setup-and-libraries}}

\begin{Shaded}
\begin{Highlighting}[]
\FunctionTok{library}\NormalTok{(magrittr)}
\FunctionTok{library}\NormalTok{(lubridate) }\CommentTok{\#ymd\_hms}
\FunctionTok{library}\NormalTok{(tidyverse)}
\FunctionTok{library}\NormalTok{(kableExtra) }\CommentTok{\#kable}
\end{Highlighting}
\end{Shaded}

\hypertarget{introduction}{%
\section{Introduction}\label{introduction}}

This Code Clinic problem is about calculating statistics from a data
set. It's easy stuff, but presents a good example of how different
languages accomplish common tasks.

\hypertarget{import-the-source-data}{%
\section{Import the source data}\label{import-the-source-data}}

The data set is weather data captured from Lake Pend O'Reille in
Northern Idaho --- \url{https://github.com/lyndadotcom/LPO_weatherdata}.
We have almost 20 megabytes of data from the years 2012 thorough 2015.
That data is available in the folder with other exercise files. Each
observation in the data includes several variables and the data is
straightforward.

\begin{Shaded}
\begin{Highlighting}[]
\NormalTok{mytempfile }\OtherTok{\textless{}{-}} \FunctionTok{tempfile}\NormalTok{()}

\NormalTok{readOneFile }\OtherTok{\textless{}{-}} \ControlFlowTok{function}\NormalTok{(dataPath) \{}
  \FunctionTok{read.table}\NormalTok{(dataPath,}
             \AttributeTok{header =} \ConstantTok{TRUE}\NormalTok{,}
             \AttributeTok{stringsAsFactors =} \ConstantTok{FALSE}\NormalTok{)}
\NormalTok{\}}
\end{Highlighting}
\end{Shaded}

With the large file, we should create the progress bar to see how long
we should know to wait for the reading into r by using
\texttt{txtProgressBar} function.

\begin{Shaded}
\begin{Highlighting}[]
\NormalTok{myProgressBar }\OtherTok{\textless{}{-}} \FunctionTok{txtProgressBar}\NormalTok{(}\AttributeTok{min =} \DecValTok{2012}\NormalTok{, }\AttributeTok{max =} \DecValTok{2015}\NormalTok{, }\AttributeTok{style =} \DecValTok{3}\NormalTok{)}
\end{Highlighting}
\end{Shaded}

\begin{Shaded}
\begin{Highlighting}[]
\ControlFlowTok{for}\NormalTok{ (dataYear }\ControlFlowTok{in} \DecValTok{2012}\SpecialCharTok{:}\DecValTok{2015}\NormalTok{) \{}
  
\NormalTok{  dataPath }\OtherTok{\textless{}{-}}
    \FunctionTok{paste0}\NormalTok{(}
\NormalTok{      link,}
\NormalTok{      dataYear,}
      \StringTok{".txt"}\NormalTok{)}
  
  \ControlFlowTok{if}\NormalTok{ (}\FunctionTok{exists}\NormalTok{(}\StringTok{"LPO\_weather\_data"}\NormalTok{)) \{}
\NormalTok{    mytempfile }\OtherTok{\textless{}{-}} \FunctionTok{readOneFile}\NormalTok{(dataPath)}
\NormalTok{    LPO\_weather\_data }\OtherTok{\textless{}{-}} \FunctionTok{rbind}\NormalTok{(LPO\_weather\_data, mytempfile)}
\NormalTok{  \} }\ControlFlowTok{else}\NormalTok{ \{}
\NormalTok{    LPO\_weather\_data }\OtherTok{\textless{}{-}} \FunctionTok{readOneFile}\NormalTok{(dataPath)}
\NormalTok{  \}}
  \FunctionTok{setTxtProgressBar}\NormalTok{(myProgressBar, }\AttributeTok{value =}\NormalTok{ dataYear)}
  
\NormalTok{\}}
\end{Highlighting}
\end{Shaded}

\begin{verbatim}
##   |                                                                              |                                                                      |   0%  |                                                                              |=======================                                               |  33%  |                                                                              |===============================================                       |  67%  |                                                                              |======================================================================| 100%
\end{verbatim}

\hypertarget{confirm-the-results-of-the-import}{%
\section{Confirm the results of the
import}\label{confirm-the-results-of-the-import}}

\begin{Shaded}
\begin{Highlighting}[]
\FunctionTok{kbl}\NormalTok{(}\FunctionTok{head}\NormalTok{(LPO\_weather\_data, }\AttributeTok{n =} \DecValTok{3}\NormalTok{), }\AttributeTok{booktabs =}\NormalTok{ T) }\SpecialCharTok{\%\textgreater{}\%} 
  \FunctionTok{kable\_styling}\NormalTok{(}\AttributeTok{latex\_options =} \FunctionTok{c}\NormalTok{(}\StringTok{"striped"}\NormalTok{, }\StringTok{"scale\_down"}\NormalTok{))}
\end{Highlighting}
\end{Shaded}

\begin{table}
\centering
\resizebox{\linewidth}{!}{
\begin{tabular}[t]{llrrrrrrr}
\toprule
date & time & Air\_Temp & Barometric\_Press & Dew\_Point & Relative\_Humidity & Wind\_Dir & Wind\_Gust & Wind\_Speed\\
\midrule
\cellcolor{gray!6}{2012\_01\_01} & \cellcolor{gray!6}{00:02:14} & \cellcolor{gray!6}{34.3} & \cellcolor{gray!6}{30.5} & \cellcolor{gray!6}{26.9} & \cellcolor{gray!6}{74.2} & \cellcolor{gray!6}{346.4} & \cellcolor{gray!6}{11} & \cellcolor{gray!6}{3.6}\\
2012\_01\_01 & 00:08:29 & 34.1 & 30.5 & 26.5 & 73.6 & 349.0 & 12 & 8.0\\
\cellcolor{gray!6}{2012\_01\_01} & \cellcolor{gray!6}{00:14:45} & \cellcolor{gray!6}{33.9} & \cellcolor{gray!6}{30.6} & \cellcolor{gray!6}{26.8} & \cellcolor{gray!6}{75.0} & \cellcolor{gray!6}{217.8} & \cellcolor{gray!6}{12} & \cellcolor{gray!6}{9.2}\\
\bottomrule
\end{tabular}}
\end{table}

\begin{Shaded}
\begin{Highlighting}[]
\FunctionTok{kbl}\NormalTok{(}\FunctionTok{tail}\NormalTok{(LPO\_weather\_data, }\AttributeTok{n =} \DecValTok{3}\NormalTok{), }\AttributeTok{booktabs =}\NormalTok{ T) }\SpecialCharTok{\%\textgreater{}\%} 
  \FunctionTok{kable\_styling}\NormalTok{(}\AttributeTok{latex\_options =} \FunctionTok{c}\NormalTok{(}\StringTok{"striped"}\NormalTok{, }\StringTok{"scale\_down"}\NormalTok{))}
\end{Highlighting}
\end{Shaded}

\begin{table}
\centering
\resizebox{\linewidth}{!}{
\begin{tabular}[t]{lllrrrrrrr}
\toprule
  & date & time & Air\_Temp & Barometric\_Press & Dew\_Point & Relative\_Humidity & Wind\_Dir & Wind\_Gust & Wind\_Speed\\
\midrule
\cellcolor{gray!6}{315463} & \cellcolor{gray!6}{2015\_06\_04} & \cellcolor{gray!6}{01:04:21} & \cellcolor{gray!6}{57.7} & \cellcolor{gray!6}{29.95} & \cellcolor{gray!6}{51.22} & \cellcolor{gray!6}{79.0} & \cellcolor{gray!6}{179.41} & \cellcolor{gray!6}{9} & \cellcolor{gray!6}{6.8}\\
315464 & 2015\_06\_04 & 01:06:59 & 57.7 & 29.95 & 51.28 & 79.2 & 167.78 & 11 & 8.8\\
\cellcolor{gray!6}{315465} & \cellcolor{gray!6}{2015\_06\_04} & \cellcolor{gray!6}{01:09:21} & \cellcolor{gray!6}{57.7} & \cellcolor{gray!6}{29.95} & \cellcolor{gray!6}{51.22} & \cellcolor{gray!6}{79.0} & \cellcolor{gray!6}{163.40} & \cellcolor{gray!6}{12} & \cellcolor{gray!6}{10.0}\\
\bottomrule
\end{tabular}}
\end{table}

\begin{Shaded}
\begin{Highlighting}[]
\FunctionTok{print}\NormalTok{(}\FunctionTok{paste}\NormalTok{(}\StringTok{"Number of rows imported: "}\NormalTok{, }\FunctionTok{nrow}\NormalTok{(LPO\_weather\_data)))}
\end{Highlighting}
\end{Shaded}

\begin{verbatim}
## [1] "Number of rows imported:  315465"
\end{verbatim}

\hypertarget{calculate-the-coefficient-of-barometric-pressure}{%
\section{Calculate the Coefficient of Barometric
Pressure}\label{calculate-the-coefficient-of-barometric-pressure}}

The problem is simple: Write a function that accepts \ldots{} a
beginning date and time \ldots and\ldots{} an ending date and
time\ldots{}

\begin{Shaded}
\begin{Highlighting}[]
\NormalTok{startDateTime }\OtherTok{\textless{}{-}} \StringTok{"2014{-}01{-}02 12:03:34"}
\NormalTok{endDateTime }\OtherTok{\textless{}{-}} \StringTok{"2014{-}01{-}04 12:03:34"}
\end{Highlighting}
\end{Shaded}

\ldots then\ldots{} inclusive of those dates and times return the
coefficient of the slope of barometric pressure.

helper function to get a subset of LPO\_weather\_data observations are
the date range variables are barometric pressure, date, and time

\begin{quote}
Transform dates stored as character or numeric vectors to POSIXct
objects. The ymd\_hms() family of functions recognizes all
non-alphanumeric separators (with the exception of ``.'' if frac = TRUE)
and correctly handles heterogeneous date-time representations. For more
flexibility in treatment of heterogeneous formats, see low level parser
parse\_date\_time().
\end{quote}

\begin{Shaded}
\begin{Highlighting}[]
\NormalTok{getBaromPressures }\OtherTok{\textless{}{-}} \ControlFlowTok{function}\NormalTok{(dateTimeInterval) \{}
  \FunctionTok{subset}\NormalTok{(LPO\_weather\_data,}
         \FunctionTok{ymd\_hms}\NormalTok{(}\FunctionTok{paste}\NormalTok{(date, time)) }\SpecialCharTok{\%within\%}\NormalTok{ dateTimeInterval,}
         \AttributeTok{select =} \FunctionTok{c}\NormalTok{(Barometric\_Press, date, time)}
\NormalTok{         )}
\NormalTok{\}}

\NormalTok{calculateBaroPress }\OtherTok{\textless{}{-}} \ControlFlowTok{function}\NormalTok{(startDateTime, endDateTime) \{}
\NormalTok{  dateTimeInterval }\OtherTok{\textless{}{-}} \FunctionTok{interval}\NormalTok{(}\FunctionTok{ymd\_hms}\NormalTok{(startDateTime),}
                               \FunctionTok{ymd\_hms}\NormalTok{(endDateTime))}
  
\NormalTok{  baroPress }\OtherTok{\textless{}{-}} \FunctionTok{getBaromPressures}\NormalTok{(dateTimeInterval)}
  
\NormalTok{  slope }\OtherTok{\textless{}{-}} \FunctionTok{ymd\_hms}\NormalTok{(}\FunctionTok{paste}\NormalTok{(baroPress}\SpecialCharTok{$}\NormalTok{date, baroPress}\SpecialCharTok{$}\NormalTok{time))}
  
  \FunctionTok{lm}\NormalTok{(Barometric\_Press }\SpecialCharTok{\textasciitilde{}}\NormalTok{ slope, }\AttributeTok{data =}\NormalTok{ baroPress)}

\NormalTok{\}}

\FunctionTok{calculateBaroPress}\NormalTok{(startDateTime, endDateTime)}
\end{Highlighting}
\end{Shaded}

\begin{verbatim}
## 
## Call:
## lm(formula = Barometric_Press ~ slope, data = baroPress)
## 
## Coefficients:
## (Intercept)        slope  
##  -3.090e+03    2.245e-06
\end{verbatim}

A rising slope indicates an increasing barometric pressure, which
typically means fair and sunny weather.

\begin{figure}
\centering
\includegraphics{Barometric_rising slope.PNG}
\caption{Barometric \textasciitilde{} rising slope (adapted from
LinkedIn Learning)}
\end{figure}

A falling slope indicates a decreasing barometric pressure, which
typically means stormy weather.

\begin{figure}
\centering
\includegraphics{Barometric_falling slope.PNG}
\caption{Barometric \textasciitilde{} falling slope (adapted from
LinkedIn Learning)}
\end{figure}

We're only asking for the coefficient -- but some may choose to generate
a graph of the results as well.

\hypertarget{graph-barometric-pressure}{%
\section{Graph Barometric Pressure}\label{graph-barometric-pressure}}

\begin{Shaded}
\begin{Highlighting}[]
\NormalTok{graphBaroPressure }\OtherTok{\textless{}{-}} \ControlFlowTok{function}\NormalTok{(startDateTime, endDateTime ) \{}
  
\NormalTok{  dateTimeInterval }\OtherTok{\textless{}{-}} \FunctionTok{interval}\NormalTok{(}\FunctionTok{ymd\_hms}\NormalTok{(startDateTime),}
                               \FunctionTok{ymd\_hms}\NormalTok{(endDateTime))}
  
\NormalTok{  baroPress }\OtherTok{\textless{}{-}} \FunctionTok{getBaromPressures}\NormalTok{(dateTimeInterval)}
  
\NormalTok{  thisDateTime }\OtherTok{\textless{}{-}} \FunctionTok{ymd\_hms}\NormalTok{(}\FunctionTok{paste}\NormalTok{(baroPress}\SpecialCharTok{$}\NormalTok{date, baroPress}\SpecialCharTok{$}\NormalTok{time))}
  
  \FunctionTok{plot}\NormalTok{(}
    \AttributeTok{x =}\NormalTok{ thisDateTime,}
    \AttributeTok{y =}\NormalTok{ baroPress}\SpecialCharTok{$}\NormalTok{Barometric\_Press,}
    \AttributeTok{xlab =} \StringTok{"Date and Time"}\NormalTok{,}
    \AttributeTok{ylab =} \StringTok{"Barometric Pressure"}\NormalTok{,}
    \AttributeTok{main =} \FunctionTok{paste}\NormalTok{(}
      \StringTok{"Barometric Pressure from "}\NormalTok{,}
      \FunctionTok{ymd\_hms}\NormalTok{(startDateTime),}
      \StringTok{"to"}\NormalTok{,}
      \FunctionTok{ymd\_hms}\NormalTok{(endDateTime)}
\NormalTok{    )}
\NormalTok{  )}
  \FunctionTok{abline}\NormalTok{(}\FunctionTok{calculateBaroPress}\NormalTok{(startDateTime, endDateTime), }\AttributeTok{col =} \StringTok{"red"}\NormalTok{)}
\NormalTok{\}}

\FunctionTok{graphBaroPressure}\NormalTok{(startDateTime, endDateTime)}
\end{Highlighting}
\end{Shaded}

\includegraphics{C:/Users/minhh/Box Sync/R/Master R for Data Science/Code-Clinic-R/docs/summarize-theweather_files/figure-latex/unnamed-chunk-10-1.pdf}

\end{document}
